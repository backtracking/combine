\pdfminorversion 7
\pdfobjcompresslevel 3

\documentclass[a4paper]{article}
\special{papersize=210mm,297mm}
\usepackage[utf8]{inputenc}
\usepackage[T1]{fontenc}
\usepackage{cite}
\usepackage[francais]{babel}
\usepackage[bookmarks=false,colorlinks,linkcolor=blue]{hyperref}
\usepackage[top=3cm,bottom=2cm,left=3cm,right=2cm]{geometry}
\usepackage{graphicx}
\usepackage{wrapfig}
\usepackage{subfig}
\usepackage{eso-pic}
\usepackage{array}
\usepackage{color}
\usepackage[table]{xcolor}
\usepackage{url}
\usepackage{listings}
\usepackage{eurosym}
\usepackage{url}
\usepackage{textcomp}
\usepackage{fancyhdr} 

\definecolor{lightgray}{gray}{0.9}

\title{Rapport de Stage, Licence Informatique 3\up{ème} Année}
\author{Rémy \textsc{EL-SIBAIE BESOGNET}}

\newcommand{\HRule}{\rule{\linewidth}{0.5mm}}


\begin{document}


\begin{titlepage}

\begin{center}


% Upper part of the page

\textsc{\LARGE Université Paris Sud}\\[1.5cm]

\textsc{\Large Rapport de stage}\\[0.5cm]

\textsc{\Large Licence 3 Informatique}\\[0.5cm]

% Title
\HRule \\[0.4cm]
{ \huge \bfseries OCaml et Combinatoire}\\[0.4cm]

\HRule \\[1.5cm]

% Author and supervisor
\begin{minipage}{0.4\textwidth}
\begin{flushleft} \large
\emph{Auteur:}\\
Rémy \textsc{EL SIBAÏE BESOGNET}
\end{flushleft}
\end{minipage}
\begin{minipage}{0.4\textwidth}
\begin{flushright} \large
\emph{Encadrant:} \\
Jean-Christophe \textsc{FILLIÂTRE}
\end{flushright}
\end{minipage}

\vfill

% Bottom of the page
{\large \today}

\end{center}

\end{titlepage}


~
\vfill

\begin{center}
\section*{Résumé}
\end{center}
En informatique, de nombreux problèmes de combinatoire tels que le problème des 
n-reines ou les problèmes de recouvrement de tuiles peuvent se formaliser d'une
façon commune :
Le recouvrement exact de matrice, soit EMC\footnote{Exact Matrix Covering}. 
Plusieurs algorithmes permettent alors 
de résoudre cette question et donc les problèmes ci-dessus de façon générique.

Le but de mon stage est d'implémenter, en Ocaml, deux algorithmes de
résolution du recouvrement exact de matrice ayant pour nom 
Zero-suppressed Binary Decision
Diagram et Dancing Links. 
Ensuite, l'objectif est d'apporter des méthodes 
simplifiant la modélisation des problèmes de combinatoire pour les ramener
à un probleme de type EMC. L'utilisateur pourra par 
exemple utiliser un mini langage pour décrire son problème.

Pour finir, cette bibliothèque doit être organisée de façon à être soumise à la 
Caml-list. Ainsi, la communauté Caml pourra en bénéficier, selon son utilité.

\vfill




\newpage
\section*{Remerciements}

\newpage
\tableofcontents
\newpage
\listoffigures

\newpage
\section{Présentation}


J'ai effectué ce stage dans le cadre de ma troisième année de Licence en 
Informatique, sous la direction de Jean-Christophe Filliâtre (Equipe ProVal, 
LRI, Université Paris Sud 11 et CNRS / INRIA Saclay - Île-de-France). L'équipe 
ProVal a pour objectif de présenter des outils permettant de démontrer la 
validité de programmes par rapport à un comportement attendu en utilisant des
méthodes formelles de preuve.
Ces outils sont en majorité dévellopés en langage OCaml. 

Le but de mon stage n'est donc pas concentré sur le coeur de metier de 
l'équipe. Il correspond plutôt à une contribution au langage OCaml. 
Cette contribution
est sous la forme d'une bibliothèque OCaml pour la combinatoire, apportant
des méthodes simple permettant de modéliser des problèmes de combinatoire
sous forme de matrice de contraintes. Elle inclue aussi l'implémentation de 
deux algorithmes permettant de résoudre le problème du recouvrement exact de
matrice.


\section{Recouvrement Exact de Matrices}

\subsection{Définition}

Résoudre le problème de recouvrement exact de matrice 
(ou EMC~\footnote{Exact Matrix Covering en Anglais}) revient à,
partant d'une matrice de booléens donnée, trouver un sous ensemble de 
lignes de cette matrice tel qu'il n'y ai qu'un et un seul '1' (ou \emph{true}) 
par colonne.
On considère par exemple la matrice suivante, de taille 4x5 avec les lignes
notées de 'a' à 'e' : 

\[
  \begin{array}{ c }
   a. \\
   b. \\
   c. \\
   d. \\
   e. \\
  \end{array}
\left(
  \begin{array}{ c c c c c }
   1 & 0 & 1 & 1 \\
   0 & 1 & 1 & 0 \\
   1 & 1 & 0 & 1 \\
   1 & 0 & 0 & 1 \\
   0 & 1 & 0 & 0
  \end{array} \right)
\]

Dans cette exemple, on choisit alors differents sous ensemble et on vérifie 
s'ils correspondent à une solution :

\begin{description}
\item[$ {d, e} $] ne correspond pas à une solution du probleme (pas de 1 dans la 3\up{e} colonne)
\item[$ {a, e} $] constitue bien une solution (un et un seul 1 par colonne)
\item[$ {b, d} $] de même
\end{description}

Il faut comprendre que le problème EMC correspond en fait à une liste de 
contraintes. Le but est donc de choisir un ensemble de cas qui satisfaint 
chaque contrainte une seule et une seule fois.
Ce problème est connu pour avoir une complexité importante. On pourra observer 
plus tard qu'il peut être résolu par \emph{backtracking}.



\subsection{Applications}

Résoudre EMC est finalement une forme de programmation par contrainte. 
Plusieurs problèmes peuvent donc être encapsulés dans EMC à 
condition de pouvoir les modéliser tels quels. Je donnerais ici quelques 
exemples connus.


\subsubsection{Pavage}
Un problème de pavage est la recherche du nombre de façon de recouvrir un
plateau de jeu avec des tuiles (c'est à dire des pièces) d'une taille et d'une
forme choisie. 
Ces problème correspondent parfaitement à la situation
et sont relativement simples à représenter. 
Effectivement dans un problème
de pavage on cherche généralement à recouvrir chaque case une seule et une 
seule fois. Ce qui ressemble étrangement à la règle d'un problème EMC.

\newpage
On prend le problème trivial suivant : 

\emph{De combien de façon peut-on recouvrir un échiquier (taille 8x8) par des 
des dominos ?}
\\

\begin{figure}
\centering
\rowcolors{1}{lightgray}{lightgray}
\caption{\label{chessboard4x4} Echiquier 4x4 et dominos}
\[
  \begin{array}{|c|c|c|c|}
		\hline
   	0 & 1 & 2 & 3 \\
		\hline
    4	& 5 & 6 & 7 \\
		\hline
   	8 & 9 & 10 & 11 \\
		\hline
   	12 & 13 & 14 & 15 \\
		\hline
\end{array}
	\textrm{ }
\begin{array}{ |c| }
	\hline
		\\
	\hline
    \\
	\hline
  \end{array}
	\textrm{ }
\begin{array}{ |c|c| }
	\hline
		& \\ 
	\hline
  \end{array}
\]
\end{figure}

Ce problème est trivial puisqu'il existe une formule mathématique pour le 
résoudre. On en connait donc facilement les solutions. De plus, on ne prend 
pas en compte la quantité de domino que l'on peut poser ni les symétries 
éventuelles du problème. On construit alors la
matrice EMC correspondante. Une colonne de la matrice représente une case de l'éc
hiquier et une ligne de cette même matrice représente une façon de poser 
une tuile. On numérote donc les cases de 0 à 15.

Par exemple, poser un domino horizontalement sur la sur la case 0 donne la ligne
de matrice suivante : 

\[
  \begin{array}{ c c c c c c c c c c c c c c c c }
	1 & 1 & 0 & 0 & 0 & 0 & 0 & 0 & 0 & 0 & 0 & 0 & 0 & 0 & 0 & 0 
  \end{array}
\]

On continue l'opération pour chaque case et on ajoute la ligne correspondante
si la pièce peut être posée à cet endroit. 
Une fois que toutes les cases du plateau on été testées pour chacune des pièces,
on obtient la matrice EMC résultat. On a bien modélisé un problème de pavage 
sous forme de problème EMC dont nous verrons les algorithmes de résolution par
la suite.

\subsubsection{Sudoku}

Il n'est pas indispensable de décrire à nouveau le principe du 
Sudoku~\cite{sudoku} qui est massivement connu. Mais une explication concernant
sa modélisation en EMC s'impose.
Pour les problèmes de pavage, la contrainte se situait au niveau des cases. Par
contre, en Sudoku, mettre une valeur dans une case du quadrillage impose et
dépend de 5 contraintes sur la valeur et l'espace : 
\begin{itemize}
\item la case (l'espace)
\item la ligne (la valeur)
\item la colonne (la valeur)
\item la cellule (la valeur)
\end{itemize}


\begin{figure}
\centering
\rowcolors{1}{lightgray}{lightgray}
\caption{\label{sudoku} Echiquier 4x4 et dominos}
\[
  \begin{array}{|c|c|c|c|}
		\hline
   	0 & 1 & 2 & 3 \\
		\hline
    4	& 5 & 6 & 7 \\
		\hline
   	8 & 9 & 10 & 11 \\
		\hline
   	12 & 13 & 14 & 15 \\
		\hline
\end{array}
\end{figure}

5 & 3 &  &  & 7 &  &  &  & 
6&0&0&1&9&5&0&0&0
0&9&8&0&0&0&0&6&0
8&0&0&0&6&0&0&0&3
4&0&0&8&0&3&0&0&1
7&0&0&0&2&0&0&0&6
0&6&0&0&0&0&2&8&0
0&0&0&4&1&9&0&0&5
0&0&0&0&8&0&0&7&9


Effectivement, une fois qu'un élément est mit dans une case, il utilise l'espace
de cette case et empêche n'importe quel élément de prendre sa position. 
Par contre, il empêche aussi qu'un élément de même \emph{valeur} soit mis dans 
sa colonne, sa ligne et sa cellule (sous entendu les grosses cases de taille 
3x3). Il faut donc représenter chacune de ces contraintes dans les lignes de 
la futur matrice EMC. 

\subsubsection{N-reines}

\section{Recherche des solutions}

Deux algorithmes de backtracking

\subsection{La structure Zero-Supressed Binary Decision Diagram}

\subsection{L'algorithme Dancing Links}

DLX~\cite{dlx}

\subsection{ZDD vs DLX}


\section{Une bibliothèque OCaml}

\subsection{Architecture}

\subsection{Un mini langage pour les problèmes de pavage}

 expliquer optimisation des symétries ici

\section{Conclusion}


\bibliographystyle{plain}
\bibliography{./biblio}


\end{document}
