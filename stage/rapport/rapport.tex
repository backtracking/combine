\pdfminorversion 7
\pdfobjcompresslevel 3

\documentclass[a4paper]{article}
\special{papersize=210mm,297mm}
\usepackage[utf8]{inputenc}
\usepackage[T1]{fontenc}
\usepackage{cite}
\usepackage[francais]{babel}
\usepackage[bookmarks=false,colorlinks,linkcolor=blue]{hyperref}
\usepackage[top=3cm,bottom=2cm,left=3cm,right=2cm]{geometry}
\usepackage{graphicx}
\usepackage{wrapfig}
\usepackage{subfig}
\usepackage{eso-pic}
\usepackage{url}
\usepackage{listings}
\usepackage{eurosym}
\usepackage{url}
\usepackage{textcomp}
\usepackage{fancyhdr} 

\title{Rapport de Stage, Licence Informatique 3\up{ème} Année}
\author{Rémy \textsc{EL-SIBAIE BESOGNET}}

\newcommand{\HRule}{\rule{\linewidth}{0.5mm}}


\begin{document}


\begin{titlepage}

\begin{center}


% Upper part of the page

\textsc{\LARGE Université Paris Sud}\\[1.5cm]

\textsc{\Large Rapport de stage}\\[0.5cm]

\textsc{\Large Licence 3 Informatique}\\[0.5cm]

% Title
\HRule \\[0.4cm]
{ \huge \bfseries OCaml et Combinatoire}\\[0.4cm]

\HRule \\[1.5cm]

% Author and supervisor
\begin{minipage}{0.4\textwidth}
\begin{flushleft} \large
\emph{Auteur:}\\
Rémy \textsc{EL SIBAÏE BESOGNET}
\end{flushleft}
\end{minipage}
\begin{minipage}{0.4\textwidth}
\begin{flushright} \large
\emph{Encadrant:} \\
Jean-Christophe \textsc{FILLIÂTRE}
\end{flushright}
\end{minipage}

\vfill

% Bottom of the page
{\large \today}

\end{center}

\end{titlepage}


~
\vfill

\begin{center}
\section*{Résumé}
\end{center}
En informatique, de nombreux problèmes de combinatoire tels que le problème des 
n-reines ou les problèmes de recouvrement de tuiles peuvent se formaliser d'une
façon commune :
Le recouvrement exact de matrice, soit EMC\footnote{Exact Matrix Covering}. 
Plusieurs algorithmes permettent alors 
de résoudre cette question et donc les problèmes ci-dessus de façon générique.

Le but de mon stage est d'implémenter, en Ocaml, deux algorithmes de
résolution du recouvrement exact de matrice ayant pour nom 
Zero-suppressed Binary Decision
Diagram et Dancing Links. 
Ensuite, l'objectif est d'apporter des méthodes 
simplifiant la formalisation des problèmes de combinatoire pour les ramener
à un probleme de type EMC. L'utilisateur pourra par 
exemple utiliser un mini langage pour décrire son problème.

Pour finir, cette bibliothèque doit être organisée de façon à être soumise à la 
Caml-list. Ainsi, la communauté Caml pourra en bénéficier, selon son utilité.

\vfill




\newpage
\section*{Remerciements}

\newpage
\tableofcontents
\newpage
\listoffigures

\newpage
\section{Présentation}

contexte du stage

\section{Recouvrement Exact de Matrices}

\subsection{Définition}

\subsection{Applications}

\subsubsection{Pavage}
\subsubsection{Sudoku}
\subsubsection{N-reines}

\section{Recherche des solutions}

Deux algorithmes de backtracking

\subsection{La structure Zero-Supressed Binary Decision Diagram}

\subsection{L'algorithme Dancing Links}

DLX~\cite{dlx}

\subsection{ZDD vs DLX}


\section{Une bibliothèque OCaml}

\subsection{Architecture}

\subsection{Un mini langage pour les problèmes de pavage}

 expliquer optimisation des symétries ici

\section{Conclusion}


\bibliographystyle{plain}
\bibliography{./biblio}


\end{document}
